%CS-113 S18 HW-2
%Released: 2-Feb-2018
%Deadline: 16-Feb-2018 7.00 pm
%Authors: Abdullah Zafar, Emad bin Abid, Moonis Rashid, Abdul Rafay Mehboob, Waqar Saleem.


\documentclass[addpoints]{exam}

% Header and footer.
\pagestyle{headandfoot}
\runningheadrule
\runningfootrule
\runningheader{CS 113 Discrete Mathematics}{Homework II}{Spring 2018}
\runningfooter{}{Page \thepage\ of \numpages}{}
\firstpageheader{}{}{}

\boxedpoints
\printanswers
\usepackage[table]{xcolor}
\usepackage{amsfonts,graphicx,amsmath,hyperref}
\title{Habib University\\CS-113 Discrete Mathematics\\Spring 2018\\HW 2}
\author{$ms03977$}  % replace with your ID, e.g. oy02945
\date{Due: 19h, 16th February, 2018}


\begin{document}
\maketitle

\begin{questions}



\question

%Short Questions (25)

\begin{parts}

 
  \part[5] Determine the domain, codomain and set of values for the following function to be 
  \begin{subparts}
  \subpart Partial
  \subpart Total
  \end{subparts}

  \begin{center}
    $y=\sqrt{x}$
  \end{center}

  \begin{solution}
    
    
    i) For the function to be partial, it is necessary for some values of the domain to have an image in the co-domain, while the other values don't have an image in the co-domain. Such that if $y=\sqrt{x}$ is a function with domain as the set of real numbers R with images formed in the co-domain also belonging to the set of real numbers R, the function will be partial as it has no image in it's co-domain for negative real numbers taken as input but has images for positive real numbers.
    
    For example
    \begin{center}
        $f(x) =\sqrt{x}$ if x \in R 
    
    
        $f(-1) =\sqrt{-1} = i(iota) $,which is a complex number and is not present in the co-domain of function $f(x) =\sqrt{x}$
    
    
    
    \end{center}
    
    ii.) But, if we restrict the domain to have values which are in the set of positive real numbers only, the function will transform into a total function which is a function for which each value in it's domain has an image in the co-domain.
    
    
    \begin{center}
        $f(x) =\sqrt{x}$ if x \in R+
        
    \end{center}
  \end{solution}
  
  \part[5] Explain whether $f$ is a function from the set of all bit strings to the set of integers if $f(S)$ is the smallest $i \in \mathbb{Z}$� such that the $i$th bit of S is 1 and $f(S) = 0$ when S is the empty string. 
  
  \begin{solution}
    
    
    $f$(S) is not a function from set of all bit strings to the set of integers because it doesn't meet the requirement of being a function, as for a function, it is first made sure that the function is defined for which type of values as it is stated in thee question that f is a function from set the set of all bit strings to the set of integers if $f$(S) is smallest then i=1 and f(S)=0. but there isn't even an image for any string that is 0 for example 00 or 000.
    So. the function is not defined for $f$(000) and thus is not a function.
    
    
    
    
    
  \end{solution}

  \part[15] For $X,Y \in S$, explain why (or why not) the following define an equivalence relation on $S$:
  \begin{subparts}
    \subpart ``$X$ and $Y$ have been in class together"
    \subpart ``$X$ and $Y$ rhyme"
    \subpart ``$X$ is a subset of $Y$"
  \end{subparts}

  \begin{solution}
    Equivalence relation requires a set to have a symmetric, reflexive and transitive relation at the same time.
    
    
    i) This defines a reflexive relation because if X is present in X's class and this is a fact for Y too. [(X,X),(Y,Y)] $\in$ S. This set also fulfills the demands of being symmetric as X is present in Y's class so Y is also present in X's class. [(X,Y),(Y,X)] $\in$ S. Transitive property is not defined for this because X is in Y have been in class together and there is some other person Z who is with Y in a class together, it's not necessary for X and Z to be in the same class because it's possible that Y and Z may have been in a different class and X and Y occurring in a different one. So, if (X,Y) and (Y,Z) $\in$ S. It's not necessary that (X,Z) $\in$ S. So equivalence relation is not defined for this set.


    
    ii) This defines a reflexive relation because if X and Y rhyme, then they both rhyme with themselves too. [(X,X),(Y,Y)] $\in$ S. This set also fulfills the demands of being symmetric as X rhymes with Y so Y also rhymes with X.
    [(X,Y),(Y,X)] $\in$ S. Transitive property is also defined for this because X rhymes with Y and there exists a word Z to which Y rhymes with. If X rhymes with Y and Y rhymes with Z thus X rhymes with Z too. So, if (X,Y) and (Y,Z) $\in$ S. It's necessary that (X,Z) $\in$ S. Thus, equivalency is defined for this set.
 
 
    
    iii) This defines a reflexive relation because a set is a subset of itself. [(X,X),(Y,Y)] $\in$ S. This set also doesn't fulfill the demands of being symmetric as X is a subset of Y, it's not necessary that Y is also a subset of X. [(X,Y),(Y,X)] $\notin$ S . Thus equivalence is not possible
    \end{solution}

\end{parts}

%Long questions (75)
\question[15] Let $A = f^{-1}(B)$. Prove that $f(A) \subseteq B$.
  \begin{solution}
        If $A = f^{-1}(B)$, then it is invertible and we can write it as $B = f(A)$ or $f(A) = B$. Moreover, if an inverse is possible for the function, then we can assure that the function is bijective and obeys one-one correspondence that is for one element in the domain there exists a mapping(element) in it's co-domain or vice versa.$f(A)=B$ is bijective from A to B which deduces that f is surjective too. If a set is equal to another set, then it is also 
        true that the set is an improper subset of the other set.
        
        So,
        \begin{center}
        
        
        $f(A) \subseteq B $
            
        \end{center}
         
        
        
  \end{solution}

\question[15] Consider $[n] = \{1,2,3,...,n\}$ where $n \in \mathbb{N}$. Let $A$ be the set of subsets of $[n]$ that have even size, and let $B$ be the set of subsets of $[n]$ that have odd size. Establish a bijection from $A$ to $B$, thereby proving $|A| = |B|$. (Such a bijection is suggested below for $n = 3$) 

\begin{center}

  \begin{tabular}{ |c || c | c | c |c |}
    \hline
 A & $\emptyset$ & $\{2,3\}$ & $\{1,3\}$ & $\{1,2\}$ \\ \hline
 B & $\{3\}$ & $\{2\}$ & $\{1\}$ & $\{1,2,3\}$\\\hline
\end{tabular}
\end{center}

  \begin{solution}
    According to the table, the pattern is to add the value of n to the set B if it is not currently present in A, and if n is present in A, it is removed in set B.
    We have to prove this by establishing a bijection. In other words, we have to find out if it's surjective and injective. Let Ai and Aii be two subsets of A and considering $f(Ai)=f(Aii)$. We know that
    \begin{center}
        $A_i - \{n\} = f(A_i) = f(A_i_i) = A_i_i - \{n\}$
        
        
        OR
        
        
        $A_i \cup \{n\} = f(A_i) = f(A_i_i) = A_i_i \cup \{n\}.$
        
        
        In either case above, we can say that Ai=Aii, and so f is injective.
        
        To prove now that $f$ is surjective too, let B be the element of the set of images of $f$. if  B contains n, then B- $\{n\}$ is a subset of even size that is set towards B under $f$.  If B doesn't contain n , then $B \cap \{n\}$ is a subset of even size set towards B with respect to $f$.
        
        
        Since everything in the set of images has something in the set of domain that maps to it, $f$ is surjective.
        
        Bijection is established from A to B as it is surjective and injective both.
        The fact that one-one correspondence is established between set A and B draws the conclusion that each element is mapped to a single unique element.Thereby proving that the cardinality of both the sets are equal too.
        
        


        
        
        
    \end{center}{}
  \end{solution}
  
\question Mushrooms play a vital role in the biosphere of our planet. They also have recreational uses, such as in understanding the mathematical series below. A mushroom number, $M_n$, is a figurate number that can be represented in the form of a mushroom shaped grid of points, such that the number of points is the mushroom number. A mushroom consists of a stem and cap, while its height is the combined height of the two parts. Here is $M_5=23$:

\begin{figure}[h]
  \centering
  \includegraphics[scale=1.0]{m5_figurate.png}
  \caption{Representation of $M_5$ mushroom}
  \label{fig:mushroom_anatomy}
\end{figure}

We can draw the mushroom that represents $M_{n+1}$ recursively, for $n \geq 1$:
\[ 
    M_{n+1}=
    \begin{cases} 
      f(\textrm{Cap\_width}(M_n) + 1, \textrm{Stem\_height}(M_n) + 1, \textrm{Cap\_height}(M_n))  & n \textrm{ is even} \\
      f(\textrm{Cap\_width}(M_n) + 1, \textrm{Stem\_height}(M_n) + 1, \textrm{Cap\_height}(M_n)+1) & n \textrm{ is odd}  \\      
   \end{cases}
\]

Study the first five mushrooms carefully and make sure you can draw subsequent ones using the recurrence above.

\begin{figure}[h]
  \centering
  \includegraphics{mushroom_series.png}
  \caption{Representation of $M_1,M_2,M_3,M_4,M_5$ mushrooms}
  \label{fig:mushroom_anatomy}
\end{figure}

  \begin{parts}
    \part[15] Derive a closed-form for $M_n$ in terms of $n$.
  \begin{solution}
    According to the examples in the provided images, CAP heights in the sequences are: 1,2,2,3,3,3,4,4,4,4...=Cap_height(Mn)=floor(n/2)+1

                                                    
  and CAPwidth are 2,3,4,5,6 = Cap_width=n+1
  
  
  and Stemheight(Mn)=n-1 from the above examples
  
  
  The formula would be $$2(n-1) + \sum_{i=0}^{floor(n/2)+1} (n+1)-i $$ = Number of dots.
  
  
  
  
  

  

  \end{solution}
    \part[5] What is the total height of the $20$th mushroom in the series? 

    The total height of the mushroom will be the sum of Capheight and Stemheight .
    
    \begin{center}
        CaPHEIGHT+ Stemheight = total height
        Floor(n/2)+1 +(n-1) = total height
        
        20th total height = Floor(20/2)+1 +(20-1)
        height= 10+1 +19
        height= 30
        
    \end{center}
  

\end{parts}
\question
    The \href{https://en.wikipedia.org/wiki/Fibonacci_number}{Fibonacci series} is an infinite sequence of integers, starting with $1$ and $2$ and defined recursively after that, for the $n$th term in the array, as $F(n) = F(n-1) + F(n-2)$. In this problem, we will count an interesting set derived from the Fibonacci recurrence.
    
The \href{http://www.maths.surrey.ac.uk/hosted-sites/R.Knott/Fibonacci/fibGen.html#section6.2}{Wythoff array} is an infinite 2D-array of integers where the $n$th row is formed from the Fibonnaci recurrence using starting numbers $n$ and $\left \lfloor{\phi\cdot (n+1)}\right \rfloor$ where $n \in \mathbb{N}$ and $\phi$ is the \href{https://en.wikipedia.org/wiki/Golden_ratio}{golden ratio} $1.618$ (3 sf).

\begin{center}
\begin{tabular}{c c c c c c c c}
 \cellcolor{blue!25}1 & 2 & 3 & 5 & 8 & 13 & 21 & $\cdots$\\
 4 & \cellcolor{blue!25}7 & 11 & 18 & 29 & 47 & 76 & $\cdots$\\
 6 & 10 & \cellcolor{blue!25}16 & 26 & 42 & 68 & 110 & $\cdots$\\
 9 & 15 & 24 & \cellcolor{blue!25}39 & 63 & 102 & 165 & $\cdots$ \\
 12 & 20 & 32 & 52 & \cellcolor{blue!25}84 & 136 & 220 & $\cdots$ \\
 14 & 23 & 37 & 60 & 97 & \cellcolor{blue!25}157 & 254 & $\cdots$\\
 17 & 28 & 45 & 73 & 118 & 191 & \cellcolor{blue!25}309 & $\cdots$\\
 $\vdots$ & $\vdots$ & $\vdots$ & $\vdots$ & $\vdots$ & $\vdots$ & $\vdots$ & \color{blue}$\ddots$\\
 

\end{tabular}
\end{center}

\begin{parts}
  \part[10] To begin, prove that the Fibonacci series is countable.
 
    \begin{solution}
    A set S is countable if there exists a bijection between S and N(Set of Natural Numbers). To prove the bijection, we need to prove that it's surjective and injective that is onto and one to one function.
    In fibonacci sequence, $f(0) = 0 $ and $f(1)=1 $ when n =0 and n=1 respectively.
    while the proceeding terms are independent of the previous elements.
    Some of it's proceeding terms are.
    
    
    f2 = f1 + f0 = 1 + 0 = 1,
    
    
    f3 = f2 + f1 = 1 + 1 = 2,
    
    
    f4 = f3 + f2 = 2 + 1 = 3,
    
    
    f5 = f4 + f3 = 3 + 2 = 5,
    
    
    f6 = f5 + f4 = 5 + 3 = 8.
    
    
    The function from S to N is injective because every element of S is mapped to N.
    and any element of S is not left out.
    
    
    If we reverse the domain and co-domain of this function, every element of Natural numbers map to the elements of set S. Thus, it is bijective.
    
    Thus every element is mapped to a single unique element in N, It is proven to be countable.

    
    
    
    
    
    
  \end{solution}
  \part[15] Consider the Modified Wythoff as any array derived from the original, where each entry of the leading diagonal (marked in blue) of the original 2D-Array is replaced with an integer that does not occur in that row. Prove that the Modified Wythoff Array is countable. 

  \begin{solution}
    
  \end{solution}
\end{parts}

\end{questions}

\end{document}
